\documentclass[12pt]{article}
\usepackage[utf8]{inputenc}
\pagenumbering{arabic}
\usepackage{graphicx}
\usepackage{amstext}
\usepackage[usenames, dvipsnames]{color}
\graphicspath{ {images/} }


\begin{document}

\begin{titlepage}
    \begin{center}
    \begin{figure}
        \centering
        \includegraphics[scale=0.2]{logoPolimi.png}
        \vspace{1.5cm}
    \end{figure}

    \Huge\textbf{Software Engineering 2 Project - Travlendar+}
    \rule{12cm}{0.5pt}
    \Huge\textbf{RASD - Requirement Analysis and Specification Document}
    \today
    \end{center}
    
    \vspace{3cm}
    
    \begin{flushleft}
        \LARGE\textbf{Authors: }
        \newline\newline
        \Large\texttt{}{Francisco Cristóvão \\ Samsom Beyene}
    \end{flushleft}



\end{titlepage}

\newpage
  \tableofcontents
\newpage

\section{Introduction}

\subsection{Purpose}

The main goal of this project is to create a calendar-based application which provides the user a flexible and fully-featured calendar support that considers the travel time between meetings. With this in mind, the application will:
\begin{itemize}
\item compute and account for travel time between appointments, and prevent conflicts between them
\item support the user in his/her travels, adding automatically the travel time to the calendar between meetings, and suggesting the best travel option based on the available time
\end{itemize}

\subsection{Scope}
Travlendar+ is a calendar-based application that provides the user a convenient way of organizing his/her daily schedule, maximizing its productiveness and minimizing the worthless time of his/her day. This application was not only thought for the regular businessman/businesswoman, who travel in between meetings the whole day and have no time to spare, but also for the parents with a more regular daily schedule, who just want to get the best of their time while being able to pick their kids from school and take them to other activities, always being on time.
Of course the system will fully support the features of a regular calendar application (booking of appointments in a specific time and location), but in a "smart" way, being able to detect and warn the user if a new appointment is not feasible because it has a conflict (the start of it doesn't allow the needed travel time after the end of the last appointment). The application is meant to be used in the City of Milan, and so it will take advantage of the wide range of travel means and services already existing in the city, from public transports to shared bikes and cars. With the information gathered from those services, it will be able to suggest the best travel mean for the user to move between appointments, based on the available travel time, total cost, current weather and even user preferences.


\subsection{Definitions, Acronyms, Abbreviations}
API: Application Programming Interface
RASD: Requirements Analysis and Specification Document
MTBF: Mean Time Between Failure

\subsection{Revision History}

\subsection{Reference Documents}
Assignment document: Mandatory Project Assignments.pdf

\subsection{Document Structure}
Other than this introductory chapter, this RASD is organized in five more chapters. Chapter two is meant to provide an overview of the systems functionalities, the type of users it is meant for and the different kinds of interactions it contemplates, not only with the users themselves, but also with other systems. Some of the systems requirements are also slightly discussed in this chapter, even though they’ll be analysed in the following chapter. In the third chapter (as mentioned above) the systems requirements, attributes and constraints are analysed and discussed with the appropriate detail and depth, specifying exactly how they should be.
The fourth chapter deals with the formal analysis of the system using and Alloy model. It includes the Alloy model of the system, with a brief discussion on its purpose and on the relevance of using Alloy as a tool to validate our solution, given the problem we had to solve.
In the fifth chapter the effort spent by each of the group members is described by specifying the number of hours each member of the group worked on the development of this document.

\section{Overall Description}

\subsection{Product Perspective}
The application will need to communicate with both Public transport systems, bike or car sharing system of the city and Google maps API to find the exact status of available (active) types of transportation with their corresponding locations. These information’s helps the application to exactly locate the position of the user and meeting place so that it can assign the travel means with its calculated time. In addition, the application also retrieves the weather conditions while making some decisions. 
Since this application is data centric product it will need somewhere to store the schedules. For that a database is used. So the mobile application will communicate with the database (probably distributed database) to add, modify or view the schedule. All the database communication will go over the Internet.
    \begin{figure}[ht]
        \includegraphics[scale=0.52]{domainModel.png}
        \caption{Class Diagram of the domain}
    \label{fig:domainModel}
    \end{figure}
    
\subsection{Product Functions}
The main functionalities supported by the system will be:
\begin{itemize}
    \item Allow the user to book an appointment with a given time, location and duration
    \item Suggest to the user the best travel means for him to use in between appointments, based on his preferences, the time available and the weather
\end{itemize}

\subsection{User Characteristics}
The system aims at satisfying the needs of many different types of users. The first type is the "businessman" user, the user that is always in a tight schedule and needs a tool like our system to simplify the way he books and turns up to the appointments. The second type is the opposite of the "businessman" user. It's a user that has a less busy life and only needs the system to remind him of the appointments without having to worry about how to get there or when to leave to arrive on time. There's also a third type of user which is placed in between the two described previously: also has a tight office schedule but still has time for personal activities, and needs the system to help him reach this balance in the most efficient way.
All of this types of users are of working age (15 to 64 years old)\footnote{Definition from OECD: https://data.oecd.org/pop/working-age-population.htm} and know how to use a smartphone.

\subsection{Assumptions, dependencies and constraints}
[D1] The username must be unique.\\{}
[D2] The system will have access to the phone GPS functionality.\\{}
[D3] The phone will always have a connection to the internet (WiFi or mobile data).
[D4] The users authorize the system to provide user data (when needed) to the third-party applications it interacts with.


\section{Specific Requirements}

\subsection{External Interface Requirements}

\subsubsection{User Interfaces}

\subsubsection{Hardware Interfaces}
In the first release no Hardware Interfaces will be necessary, since the system doesn't need to interact physically with other systems.

\subsubsection{Software Interfaces}
Given the wide range of features the system offers, it will need to have several software interfaces. There has to be a way of storing all the user-related data (mainly login and preferences data). In that sense, the system will use a \textbf{MySQL API} to connect with a MySQL database server, where all the app data will be stored.\\
The system will also need different kinds of information about the users location and its surrounding, in order to compute the travel time between appointments. To support this functionality, the system will use \textbf{Google Maps Geolocation API} and \textbf{Google Maps API}. The first one will provide information about the user exact location and the second one will provide the real-time information about maps and traffic, and allow it to calculate the best route between two different appointments.\\
To have access to the current weather, the system will use \textbf{AccuWeather API}, which will provide information about the current weather either where the user is or where the user is going.
At last, the system also needs to get all sorts of information from many different travel means. In order to get this information, it will have a software interface with \textbf{ATM Milano} (public transport schedule and routes), \textbf{Drive Now API} (for the car sharing system) and \textbf{Ofo API} (for the bike sharing system).



\subsubsection{Communication Interfaces}
The communication between the the system and the other software interfaces (mentioned above in the software interface part of this document) is important because the system depends on these services. Here the communication is achieved when both the system and the software interfaces are connected to an Internet. The systems mobile application can use WiFi or mobile data but the software interfaces must be connected to the internet regardless of the type of connection they use.

\subsection{Functional Requirements}



\subsection{Performance Requirements}

\subsection{Design Constraints}
\subsubsection{Standard Compliance}
\subsubsection{Hardware Limitations}
\subsubsection{Any other Constraint}

\subsection{Software System Attributes}
\subsubsection{Reliability}
When deployed, no bugs must be detected on the system, and the system must be fully available and functional for 99,9\% of the calendar year. 


\subsubsection{Availability}
The services that the system provides must be available 24/7. Obviously, taking into account that this requirement is really hard to fulfill, very small deviations from this requirement will be considered acceptable.

\subsubsection{Security}
Users credentials and payment information are critical data that will be stored in the systems database. Since the system communicates with external systems, the security and inaccessibility (when not needed) of this data is a big concern that has to be assured. All of the data related to the appointments has to be secured too since it can be sensible information. 

\subsubsection{Maintainability}
To measure the systems maintainability a metric shall be used. This metric needs to be tracked during the development and kept as low as possible, in order to avoid high maintainability costs subsequently. It's obvious that the system must have a high maintainability, which means that the probability of performing a successful repair action within a given (necessary) time must be high (higher than 95\%).

\subsubsection{Portability}
The first release of the system will be available for the two most used mobile platforms, IOS and Android. Nonetheless, a new platform may arise, and so it's crucial that the system is built in a way which makes it easy to deploy in this new platform. The portability measure consists on the cost that deploying the system in a new platform implies, and in our case, this cost should be smaller than the cost of deploying other similar apps.

\section{Formal Analysis Using Alloy}

\section{Effort Spent}

\begin{center}
\begin{tabular}{ |l|l|l| } 
 \hline
 DATE & TASK & HOURS \\ 
 cell4 & cell5 & cell6 \\ 
 cell7 & cell8 & cell9 \\ 
 \hline
\end{tabular}
\end{center}

\section{References}
https://belitsoft.com/php-development-services/software-requirements-specification-document-example-international-standard\\
https://en.wikipedia.org/wiki/Portability_testing\\
https://en.wikipedia.org/wiki/Maintainability\\
https://en.wikipedia.org/wiki/Software_reliability_testing

\end{document}